\documentclass[14pt]{article}

\title{Introduction to Latex}
\author{Md. Emamul Haque}
\date{\today}

\usepackage{multicol}
\usepackage{multirow}

\begin{document}
    \maketitle
    \tableofcontents

    \pagebreak
    
    \section{Introduction}
    This is the introduction part of the document. \textbf{This text will appear in bold format.}

    \subsection{Section 1.1}
    This is section 1.1. \textit{This text will appear in italic format.\emph{Emphasis on this text}. This is again in italic}

    \subsubsection{Section 1.1.1}
    A section with another section. \underline{Underlined text.}

    \subsection{Section 1.2}
    Yet another subsection. \Large{Large Text}. {\huge Huge text}

    \section{Section 2}
    This is the second section. \emph{This text is within the emphasis tags.}

    \section*{Unlisted Section}
    This section will not show up in the table of contents.

    \section{List}
    In this section, we shall demonstrate ordered and unordered lists.

    \subsection{Unordered List}

    \begin{itemize}
        \item First item
        \item Second item
        \item Third item
        \begin{itemize}
            \item Nested item 1
            \item Nested item 2
        \end{itemize}
    \end{itemize}

    \subsection{Ordered List}

    \begin{enumerate}
        \item Item 1
        \item Item 2
        \begin{itemize}
            \item Nested Item A
            \item Nested Item B
        \end{itemize}
        \item Item 3
    \end{enumerate}

    \subsection{Descriptive List}

    \begin{description}
        \item[Description 1] Description
        \item[Description 2] Description
    \end{description}

    \section{Tables}

    \subsection{Basic Tables}

    \begin{tabular} {| c c c |}
        \hline
        cell1 & cell2 & cell3 \\
        \hline
        \hline
        cell4 & cell5 & cell6 \\
        \hline
        cell7 & cell8 & cell9 \\
        \hline
    \end{tabular}

    \subsection{Multi-column Table}

    \begin{tabular}{|c | c c |}
        \hline
        cell1 & cell2 & cell3 \\
        \cline{1-3}
        1 & \multicolumn{2}{c |}{Text} \\
        2 & 3 & 4 \\
        \hline
    \end{tabular}

    \subsection{Multi-row Table}

    \begin{tabular}{| c | c | c |}
         \hline
         cell1 & cell2 & cell3 \\
         \hline
         \multirow{2}{*}{Merged} & cell4 & cell5 \\
         & cell6 & cell7 \\
         \hline
    \end{tabular}

    \subsection{Multi-row and Multi-column Table}

    \begin{tabular}{| c | c | c | c |}
        \hline
        1 & 2 & 3 & 4 \\
        \cline{1-3}
        5 & 6 & 7 & 8 \\
        \hline
        \multicolumn{3}{|c|}{Merged} & 9 \\
        \hline
        \multirow{2}{*}{Rows} & 0 & 1 & 2 \\
        \cline{2-4}
        & 3 & 5 & 7 \\
        \hline
        \multicolumn{3}{|c|}{\multirow{2}{*}{Text}} & 8 \\
        \multicolumn{3}{|c|}{} & 7 \\
        \hline
    \end{tabular}

    \section{Math Environment}
    This is another equation:$$ E = mc^2 $$
    \begin{equation}
        E = mc^2
    \end{equation}

    \begin{equation}
        e^{\frac{a^b}{c_d}}
    \end{equation}

    Fractions:
    $\left(\frac{\frac{c}{d}}{\frac{a}{b}}\right)$

\end{document}
